\documentclass[12pt]{classnotes}
\usepackage{lipsum}
\usepackage{verbatim}

\def\tbs{\textbackslash}

\title{\vspace{-1em}Guida all'uso della \\[0.5em]classe \texttt{classnotes}\vspace{-0.5em}}
\author{Edoardo Sangiorgi}
\titleimage[scale=0.7]{logo_unife.png}

\begin{document}

\maketitle



\chapter{Introduzione}
Questo documento è una guida su come usare la classe \texttt{classnotes}.
Questa classe è pensata per facilitare la redazione di appunti delle lezioni, specialmente di corsi universitari di tipo ingegneristico (ma possono trovare utilità anche in corsi di altre discipline).

I criteri con cui è stata progettata questa classe è quella di avere degli appunti ben chiari ed eleganti, in cui la gerarchia degli argomenti sia ben definita ma anche la loro tipologia.
L'obiettivo è che sia sempre univocamente chiaro se quello che si sta leggendo sia un esempio, un approfondimento o faccia parte della trattazione principale.

Una presemma importante è che questa classe è soggetta a revisione pertanto non è detto che l'obiettivo che ci si è posti sia riuscito.
Sarà cura dell'autore provvedere ad aggiornare la classe con nuove esigenze.


\chapter{Frontespizio}
Il frontespizio è gestito in automatico dalla classe ed è definito a priori.
Si può costruire chiamando semplicemente il comando \texttt{\tbs maketitle}.

Per settare i parametri basta inserire i contenuti in \texttt{\tbs title}, \texttt{\tbs author} e \texttt{\tbs titleimage} in questo modo

\begin{verbatim}
\title{Appunti di Qualcosa}
\author{Edoardo Sangiorgi}
\titleimage[scale=0.7]{foto.png}
\end{verbatim}

Come si vede dall'esempio, è possibile passare fra parentesi tutti gli argomenti della funzione \texttt{\tbs includegraphics} tramite \texttt{\tbs titleimage}.
Se l'immagine è omessa, verrà rimpiazzata da uno spazio bianco.

\textbf{Nota}: non c'è un resize automatico delle barre orizzontali rispetto allo spazio del titolo, pertanto se si presenta la necessità di cambiarle, bisogna agire aggiungendo spazi nel comando \texttt{\tbs title} come in questo esempio
\begin{verbatim}
\title{\vspace{-0.5em}Guida all'uso della \\[0.5em]classe \texttt{classnotes}} 
\end{verbatim}



\chapter{Ambienti} % (fold)
Sono stati definiti diversi ambienti per facilitare la lettura del testo e dei suoi contenuti.
Qui sotto vengono elencati con relativi esempi.

\section{Ambienti di matematica}

\subsubsection*{Teoremi}
\begin{verbatim}
\begin{teorema}[Titolo]
    \lipsum[1]
\end{teorema}
\end{verbatim}
\begin{teorema}[Titolo]
    \lipsum[1]
\end{teorema}
Il titolo è opzionale, se non viene specificato (in tal caso non serve scrivere [\space]) non viene stampato.


\subsubsection*{Definizioni}
\begin{verbatim}
\begin{definizione}[Titolo]
    \lipsum[1]
\end{definizione}
\end{verbatim}
\begin{definizione}[Titolo]
    \lipsum[1]
\end{definizione}
Anche in questo caso il titolo è opzionale (ma consigliato).

\subsubsection*{Altri ambienti}
Riportiamo anche gli ambienti \texttt{\tbs proof}, \texttt{\tbs ricordo} e \texttt{\tbs oss} (di osservazione)
\begin{proof}
    \lipsum[1]
\end{proof}
\begin{ricordo}
    Questa cosa è importante ricordarla.
\end{ricordo}

\begin{oss}
    Questa cosa è importante osservarla
\end{oss}


\section{Ambienti di testo}
Riportiamo poi gli ambienti da inserire nel testo per evidenziare un concetto, riassumere un capitolo, approfondire un certo tema o proporre un esempio.

\subsubsection*{Nota}
\begin{verbatim}
\begin{nota}[Una certa cosa]
    Questa cosa è importante quindi le evidenziamo in questo modo
\end{nota}
\end{verbatim}
\begin{nota}[Una certa cosa]
    Questa cosa è importante quindi le evidenziamo in questo modo
\end{nota}
Il titolo può anche essere omesso.


\subsubsection*{Riassunto}
\begin{verbatim}
\begin{riassunto}[Argomento]
    \lipsum[1]
\end{riassunto}    
\end{verbatim}
\begin{riassunto}[Argomento]
    \lipsum[1]
\end{riassunto}
Il titolo può anche essere omesso.

\subsubsection*{Approfondimento}
\begin{verbatim}
\begin{focus}[Approfondimento su un certo tema]
    \lipsum{1}
\end{focus}    
\end{verbatim}
\begin{focus}[Approfondimento su un certo tema]
    \lipsum[1]
\end{focus}


\subsubsection*{Esempio}
\begin{verbatim}
\begin{esempio}[Titolo esempio]
    \lipsum[1]
\end{esempio}    
\end{verbatim}
\begin{esempio}[Titolo esempio]
    \lipsum[1]
\end{esempio}
\end{document}








