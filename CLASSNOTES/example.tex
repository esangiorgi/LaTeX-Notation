\documentclass[12pt]{classnotes}
\usepackage{lipsum}


\title{Appunti di Qualcosa}
\author{Edoardo Sangiorgi}
\titleimage[scale=0.7]{logo_unife.png}

\begin{document}

\maketitle

\tableofcontents

\chapter{Primo}
Ciao questo è un esempio di documento che utilizza la classe \texttt{classnotes}.

\lipsum[1]

\section{Sezione prima}
\lipsum[1]

\lipsum[1]
\begin{custombox}{Titolo}
\lipsum[1]
\end{custombox}

\noindent
\lipsum[1]
\lipsum[1]
\begin{dm}[Primo teorema]
    \noindent
\lipsum[1]
\end{dm}

\lipsum[1]
\begin{esempio}{}
\lipsum[1]
\end{esempio}

\lipsum[1]
\begin{appr}{}
\lipsum[1]
\end{appr}
\lipsum[1]

\chapter{Secondo}
\lipsum[1]
\begin{riassunto}{}
\lipsum[1]
\end{riassunto}
\lipsum[1]
\begin{ndr}
Questa è una nota a margine, o meglio, una nota da inserire all'interno del testo, ma che si distingue per essere più "libera" e meno formale rispetto ad altri ambienti. Può essere usata per annotazioni personali, riflessioni, o qualsiasi cosa che non rientri nelle categorie di esempio, approfondimento o riassunto.
\end{ndr}

\end{document}